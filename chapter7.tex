\chapter{THIẾT KẾ PCB}
\section{Quá trình thiết kế PCB}
\textbf{Chuyển tất cả linh kiện từ Schematic qua PCB}
\begin{figure}[H]
    \centering
    \includegraphics[width=1\textwidth]{pictures/7a.png}
\end{figure}
\begin{figure}[H]
    \centering
    \includegraphics[width=1\textwidth]{pictures/7b.png}
\end{figure}
Ở bước này cần kiểm tra nếu có linh kiện nào không chuyển được qua PCB hay gặp các lỗi thiếu footprint như bên dưới thì cần kiểm tra lại Schematic.
\begin{figure}[H]
    \centering
    \includegraphics[width=1\textwidth]{pictures/7c.png}
\end{figure}
\cleardoublepage
\textbf{Xắp xếp vị trí linh kiện}
\begin{figure}[H]
    \centering
    \includegraphics[width=0.8\textwidth]{pictures/7d.png}
\end{figure}
\begin{figure}[H]
    \centering
    \includegraphics[width=0.8\textwidth]{pictures/7e.png}
\end{figure}
\cleardoublepage
Khi sắp xếp vị trí linh kiện cần lưu ý:
\begin{itemize}
    \item Xắp xếp để đường đi dây trên toàn mạch là ngắn nhất.
    \item Các loại linh kiện nên đồng bộ với nhau (Ví dụ trong mạch dưới đây là sử các loại linh kiện dán SMD 0805)
    \item Các terminal nên bố trí ngoài rìa. 
    \item Nên xếp theo cụm chức năng.
    \item Ngoài ra cần lưu ý khoảng cách giữa các linh kiện để không bị trùng lên nhau hoặc tồn tại quá nhiều khoảng trống không cần thiết.
\end{itemize}   
